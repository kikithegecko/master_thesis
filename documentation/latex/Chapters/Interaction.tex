\chapter{Interaction}

This chapter describes the various interaction levels with the bracelet in detail and illustrates the respective algorithms.

\section{Pairing the Bracelet with a Light Source}

\section{Gesture Recognition}
%TODO explain one-dollar and three-dollar
%TODO explain 3-dimensional GSS
The most casual form of interaction with the bracelet is by drawing gestures in the air to trigger basic operations, e.g. switching a light source on or off. These gestures are recorded by the bracelet's accelerometer and processed using the ``3\$ Gesture Recognizer'' \cite{Kratz2010}, an extension of the popular ``1\$ Recognizer'' by Wobbrock et. al \cite{Wobbrock2007}. The algorithm is explained in detail in the following paragraphs.

After a gesture is recorded, it is resampled to a fixed number of points. If the gesture was drawn quickly, it would have less samples and thus less points compared to a slowly drawn gesture. In order to be able to compare these two gestures, a resampling is performed before further processing. The length of the gesture path $M$ is calculated and an increment size $I$ derived by dividing $M$ by $(N/1)$, where $N$ is the desired number of samples. The gesture path is stepped trough and after each distance $I$, a new point is inserted using linear interpolation (cf. listing \ref{lst:resample}). %TODOwe have fixed # of samples. is this necessary?

\begin{lstlisting}[label=lst:resample,language=python,frame=lt,caption=Resampling of a points path into N evenly spaced points]
def resample(points, N):
	I = path_length(points) / (N - 1)
	D = 0
	newpoints = points[:1]
	for i in range(1, len(points)):
		dist = distance(points[i-1], points[i])
		if(D + dist) > I:
			q = points[i-1] + ((I - D)/dist) * (points[i] - points[i-1])
			newpoints.append(q)
			D = distance(q, points[i])
		else:
			D = D + dist
	if(D + dist) > increment: #append last point manually
		q = newpoints[-1] + ((increment - D) / dist) * (points[i] - newpoints[-1])
		newpoints.append(q)
	return newpoints
\end{lstlisting}

In the next step, the gesture is prepared for matching against the templates by rotating it to a specific position. The so called \textit{indicative angle} $\theta$ between the gesture's centroid $C$ and its first point is calculated using the normalized scalar product of the centroid and the first point's position vectors. Afterwards the gesture is rotated so that $\theta$ is at $0\degree$. Wobbrock et. al. do this by using the inverse tangens function, however this is not possible in 3D space. Hence, \textit{Rodrigues' rotation formula} (named after French mathematician Olinde Rodrigues) was implemented instead. This efficient algorithm for rotating a vector in $\mathbb{R}^3$ takes the rotation angle $\theta$ as well as the axis unit vector $k$ as input, and calculates the following formula:

\begin{center}
\(
v_{rot} = v \cos\theta + (k \times v)\sin\theta + k (k \cdot v) (1 - \cos\theta).
\) \cite{koks2006}
\end{center}

After applying this function on the gesture, a good starting point is created for the actual recognizing part of the algorithm. Listing \ref{lst:rotate2zero} illustrates the rotation procedure.

\begin{lstlisting}[label=lst:rotate2zero,language=python,frame=lt,caption=Rotation of points so that their indicative angle is at $0 \degree$]
def rotate_to_zero(points):
	c = centroid(points)
	theta = acos(points[0] * c / (|points[0]| * |c|)
	newpoints = rotate_by(points, -theta)
	return newpoints
\end{lstlisting}

In order to harmonize gestures of different sizes, the points are scaled to a reference cube with edge length of $100$ units and translated so that the respective centroid $C$ is on the origin (cf. listing \ref{lst:scale+translate}).

\begin{lstlisting}[label=lst:scale+translate,language=python,frame=lt,caption=Scaling to reference cube and translation to origin]
def scale_and_translate(points, size): # size=100
	B = Bounding_Box(points)
	newpoints = []
	for p in points:
		q = Point()
		q.x = (p.x * (size / B.width)) - c.x
		q.y = (p.y * (size / B.height)) - c.y
		q.z = (p.z * (size / B.depth)) - c.z
		newpoints.append(q)
	return newpoints
\end{lstlisting}

After these harmonization and preparation steps, the gestures are matched against the prerecorded templates. The aforementioned steps are applied to templates as well as to newly recorded gestures, the following algorithms only apply to unrecognized gestures.

The candidate is compared to all stored templates using the average \ac{MSE} as a scoring metric. The optimal position between the candidate gesture and a stored template is possibly offset by a certain rotation, so the optimal combination of angles between the two gestures needs to be determined. Since rotations are costly in terms od computation time, the candidate gesture should be aligned to the template in as few tries as possible. Hence, as proposed by \cite{Kratz2010}, a \ac{GSS} is used to find the optimal angles $\alpha$, $\beta$, and $\gamma$ for rotation around the three axis of the coordinate system.

The \ac{GSS} algorithm was invented by US-American statistician Jack Kiefer in 1953 \cite{Kiefer1953} and relies on the Golden Ratio $\phi = 0.5(-1+\sqrt{5})$ to find the minimum value in a range.

\section{Casual Touch Input}

\section{Precise Touch Input}

\section{Presets and Configuration}

Gesture recording and recognition is triggered by a gentle double tap on the bracelet. This small but focused activation reduces unwanted triggering of the gesture recognition process, e.g. while gesturing heavily, and thus reducing false positives. The tap detection functionality is a built-in feature of the bracelet's accelerometer, configuration parameters for this process are listed in table \ref{tab:tapconf}.

In order to be recognized as a tap interaction, the initial impulse needs to be at least XX g in intensity. When calibrated like this, jerky movements like suddenly raising the hand at a high speed are correctly not recognized as a tap. However since the threshold is that high, the activation tap needs to be executed directly on the hardware which is located on the inner wrist.

The PULSE\_TMLT register configures the maximum time interval between the  impulse exceeding the threshold on the Z axis and falling back under said threshold. If the mentioned interval lasts at most $6.25 ms$, the interaction is considered as a tap.

After a tap is detected, all impulses in the following $25ms$ are ignored by the detection mechanism. This prevents bouncing effects and detecting multiple taps in a singe tap movement.

The MMA8652FC accelerometer is able to distinguish between single and double taps. The last configuration register listed in table \ref{tab:tapconf} is a parameter for double tap detection. It specifies the maximum time interval between two double taps and is set to $500 ms$, the same time interval as the Windows default between two mouse clicks of a double click \cite{doubleclick}.

\begin{table}
	\myfloatalign
	\begin{tabularx}{\textwidth}{lll} \toprule
		\tableheadline{Register Name} & \tableheadline{Parameter} & \tableheadline{Value}\\ 
		\midrule
		PULSE\_THSZ & Tap Detection Threshold & $100 g$\\ %TODO on +/-8g scale @.063g/LSB
		PULSE\_TMLT & Interval between Start and End Pulse & $6.25 ms$\\
		PULSE\_LTCY & Ignore Interval after Detection & $25 ms$\\
		PULSE\_WIND & Maximum Double Tap Interval & $500ms$ \\
		\bottomrule
	\end{tabularx}
	\caption[Tap detection configuration]{Single- and double tap detection configuration for the MMA8652FC digital accelerometer}  \label{tab:tapconf}
\end{table}