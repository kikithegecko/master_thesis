\chapter{Conclusion and Future Work}

This thesis presents the conceptualization and realization of casual interaction with a smart bracelet coupled to an ambient light source. The whole design process, from interaction conceptualization, hardware and software design to fabrication of the electronic components and implementation of the algorithms is taken into account and presented in a comprehensive manner. The focus lies on the design of interactions with a bracelet, so that frequently used tasks pose the least amount of interruption and require the least amount of focus from the user.

First impressions of potential users were positive, however a proper user study is required to confirm the acceptance of the device and the ways a user can interact with it. A possible study design would center on keeping the user distracted with different tasks while request him or her to operate the bracelet.

Revising the bracelet's software is likely required before planning any further work. The gesture recognition algorithm needs to be optimized for the processor and converted into integer arithmetic. In addition, the Bluetooth module needs some extra fine tuning to flawlessly work with the lamp. Features and optimizations like this often go beyond the scope of a strictly time-constrained thesis.

For the future, the bracelet can be evaluated in combination with several other devices. Music playback control is an example that easily comes to mind, for example. Since the hardware is designed very generic in terms of usage applications, a great part of the thesis is easily reusable. In addition, the whole design and manufacturing process was laid out in detail, so that future designers of wearable prototypes can benefit from the past experiences.

%TODO final sentence?