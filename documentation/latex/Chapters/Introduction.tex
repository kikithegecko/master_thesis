\chapter{Introduction}

Today, Smartphones are our digital assistants for of daily life, and at present, wrist-worn wearables are emerging to take the interweaving of online and offline life one step further. This poses many advantages and new possibilities, but also amplifies current issues in daily life with those gadgets. While those devices often call the attention of the user, their design around a touch-enabled screen forces the user to concentrate on the device while interacting with it. This usually distracts the user from his or her current context in daily life and forces him or her to interrupt the current task in order to interact with the device \cite{Pohl2013}.

As explained previously, interaction usually requires the full attention of the user. This doesn't suit the way we use our electronic devices in daily life. Since the user is not always able or willing to fully focus on a device, most interaction with them is perceived as annoying or interrupting, since it often interrupts all other things the user is doing at that moment. Due to physical, mental or social obstructions, which can be either temporary or permanent, a user might not be able to interact with a device as intended \cite{Pohl2013}. This can cause life with the new gadgets to be perceived as rather more frustrating than enriching.

\textit{Casual Interaction} can improve the relationship between users and devices. Limiting the required focus for a task to a suitable minimum and designing the interactions in a way that the user can decide how much effort she wants to put into it offers the flexibility needed in a life with the Internet of things. The concept is showcased in the following scenario in context of the \textit{Smart Home} and ambient light control.

\begin{quotation}
	Alice comes home from grocery shopping. While carrying her shopping to the kitchen, she quickly turns on the light by simply touching her bracelet.
	
	After storing everything, she decides to relax in the living room by reading a book. After turning the kitchen lights off by covering the bracelets with her hand, she makes herself at ease on the sofa and picks up the book. To feel more comfortable, Alice dims the living room light a little by covering the bracelet's touch surface and tilting her wrist, just as she was turning a dimmer knob. The living room lights dim accordingly, so she doesn't need to look any closer at the bracelet.
	
	Later that day, Alice prepares dinner and the dining table in the living room. She uses the bracelet's touch controls to specify the exact setup for each light in the room, picking exact colors by sliding and tapping on the bracelet. Alice saves three lighting setups together with unique activation gestures: A warm setting for their guests' arrival and enjoying the welcoming drinks, a well-matched dining setup with focus on the table and foods, and a darker, colorful mood for drinking cocktails after dinner. Changing between these presets with hand gestures allows Alice to focus on her guests and on the meal instead of wasting time and focus by fiddling with wall panels or switches.
	\end{quotation}
	
This scenario illustrates multiple levels of casual interaction with a device: A simple on/off function by covering the bracelet's touch surface does not need sight on the device or high precision, this task is easily accomplished even when the user is distracted or encumbered. Likewise eyes-free brightness dimming by covering the bracelet's capacitive surface and tilting the wrist is only slightly more complex, but this complexity is countered with increased intuition, since the interaction is closely related to the motion of turning a dimmer knob. Detailed color change by precise touch and a fine-tuned hue setup using focused touch interaction pose another level of complexity in which cognitive as well as physical attention is required. In addition, three hand gestures are available to easily access frequently used setups like bright white work light or slightly dimmed relaxing illumination.

In order to enable the interactions mentioned above, the device requires certain features. First, it should stay where it is needed without encumbering the user. Typical remotes or smartphones occupy at least one hand for every interaction they offer. This is not desired, so traditional hand-held devices do not fit the scenario presented above. Instead, there is the need for a wearable device that is attached to the body without getting in the way during everyday activities.

Second, touch-free interaction (i.e. control by gestures) should be possible to enable the above-described casual interaction scenarios. Capacitive touch input offers flexibility and versatility compared to traditional keypads and enables gesture touch input. In addition, the required hardware should be kept on a low cost level and encompass only needed components to keep energy consumption at a minimum, since every recharge procedure is cumbersome for the user.

A bracelet-type device can fulfill the requirements stated above. It is slim and doesn't encumber the user, it can be worn on the arm all day.

This thesis presents the conception, design, prototyping, manufacturing as well as programming of an interactive electronic bracelet for ambient light control in the \textit{Smart Home}. Pros and cons on different materials for wrist-worm wearables as well as a description of the incorporated electronic parts and an overview of the hardware prototyping process can be found in chapter \ref{chap:bracelet}. A short summary of an exemplary interactive light source is presented in chapter \ref{sec:lamp}, while chapter \ref{sec:interaction} focuses on casual interaction and the ways to interact with the bracelet in great detail as well as their technical implementations. Chapter \ref{chap:evaluation} summarizes the lessons learned in the process of designing and manufacturing the bracelet, as well as perceived impressions by potential users. The thesis closes with a conclusion and an outlook on possible future work in this field (chapter \ref{chap:conclusion}).