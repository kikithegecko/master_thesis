\chapter{Scenario} %TODO merge with introduction

\begin{quotation}
Alice comes home from grocery shopping. While carrying her shopping to the kitchen, she quickly turns on the light by simply touching her bracelet.

After storing everything, she decides to relax in the living room by reading a book. After turning the kitchen lights by covering the bracelets with her hand, she makes herself comfortable on the sofa and picks up the book. To feel more comfortable, Alice dims the living room light a little by touching the bracelet rim and tilting her wrist, just as she was turning a dimmer knob. The living room lights dim accordingly, so she doesn't need to look any closer at the bracelet.

Later that day, Alice prepares dinner and the dining table in the living room. She uses the bracelet's touch controls to specify the exact setup for each light in the room. Alice saves three lighting setups together with unique activation gestures: A warm setting for their guests' arrival and enjoying the welcoming drinks, a well-matched dining setup with focus on the table and foods, and a darker, colorful mood for drinking cocktails after dinner. Changing between these presets with hand gestures allows Alice to focus on her guests and on the meal instead of wasting time and focus by fiddling with wall panels or switches.
\end{quotation}

This scenario illustrates three different levels of casual interaction with a device: A simple on/off function by covering the bracelet's touch surface, intuitive and eyes-free brightness dimming by touching the bracelet and tilting the wrist, color mood change by precise touch and a fine-tuned hue setup using focused touch interaction. In addition, three hand gestures are available to easily access frequently used setups like bright white work light or slightly dimmed relaxing illumination.

In order to enable the interactions mentoined above, the device requires certain features. First, it should stay where it is needed without encumbering the user. Typical remotes or smartphones occupy at least one hand for every interaction they offer. This is not desired, so traditional hand-held devices do not fit the scenario presented above. Instead, there is the need for a wearable device that is attached to the body without getting in the way during everyday activities.

Second, touch-free interaction (i.e. control by gestures) should be possible with respect to the casual interaction scenarios pictured above. Capacitive touch input offers flexibility and versatility compared to traditional keypads and enables gestured touch input. In addition, the required hardware should be kept on a low cost level and encompass only needed components to keep energy consumption at a minimum, since every recharge procedure is cumbersome to the user.

A bracelet-typed device can fulfill the requirements stated above. It is slim and doesn't encumber the user, so it can be worn on the arm all day.