\chapter{Usage Scenario}
Alice comes home from grocery shopping. While carrying her stuff to the kitchen, she quickly turns on the light using a simple hand gesture.

After storing everything, she decides to relax in the living room by reading a book. After turning the kitchen lights off with a gesture, she makes herself comfortable on the sofa and picks up the book. To feel more comfortable, Alice dims the living room light a little by touching the bracelet rim and sliding towards her body. The living room lights dim accordingly, so she doesn't need to look any closer at the bracelet.

Later that day, Alice returns to the kitchen to prepare the dinner. While working with the meat, she notices that the light color is unfriendly in respect to her task, but her hands are too dirty to touch anything. Luckily, the light control bracelet has a built-in color sensor that she can trigger with a gesture and use to pick up the color of the meat. The working lights will then adjust to a color and brightness setting that makes her task easier.

While the food is finally in the oven, Alice returns to the living room and prepares the table for dining with her friends that will show up later. She uses the the bracelet's e-paper display and touch controls to specify exact colors and brightnesses for each light source in the room. Alice saves three lighting setups together with unique activation gestures: A colorful setting for their guests' arrival and enjoying the welcoming drinks, a well-matched dining setup with focus on the table and foods, and a darker, lounge-like mood for the after-dinner party. Changing between these presets with hand gestures allows Alice to focus on her guests and on the meal instead of wasting time and focus by fiddling with wall panels or switches.