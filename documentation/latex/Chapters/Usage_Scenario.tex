\chapter{Usage Scenario}

\begin{quotation}
Alice comes home from grocery shopping. While carrying her stuff to the kitchen, she quickly turns on the light using a simple hand gesture.

After storing everything, she decides to relax in the living room by reading a book. After turning the kitchen lights off with a gesture, she makes herself comfortable on the sofa and picks up the book. To feel more comfortable, Alice dims the living room light a little by touching the bracelet rim and sliding towards her body. The living room lights dim accordingly, so she doesn't need to look any closer at the bracelet.

Later that day, Alice prepares the dinner. While the food is finally in the oven, Alice returns to the living room and prepares the table for dining with her friends that will show up later. She uses the bracelet's e-paper display and touch controls to specify exact colors and brightnesses for each light source in the room. Alice saves three lighting setups together with unique activation gestures: A colorful setting for their guests' arrival and enjoying the welcoming drinks, a well-matched dining setup with focus on the table and foods, and a darker, lounge-like mood for the after-dinner party. Changing between these presets with hand gestures allows Alice to focus on her guests and on the meal instead of wasting time and focus by fiddling with wall panels or switches.
\end{quotation}

This scenario illustrates three different levels of casual interaction with a device: Eyes-free gestures without touching the bracelet, touch and slide interaction with optional eye contact, and focused fine-tuning of different settings with touch input and visual feedback on a display.

The mentioned interactions induce several requirements for the device. First, the device should stay where it is needed without encumbering the user. Typical remotes or smartphones occupy at least one hand for every interaction they offer. This is not desired, so traditional hand-held devices do not fit the scenario presented above. Instead, there is the need for a wearable device that is attached to the body without getting in the way during everyday activities.

Second, touch-free interaction (i.e. control by gestures) should be possible with respect to the first level of casual interaction pictured above. Capacitive touch input offers flexibility and versatility compared to traditional keypads and enables gestured touch input. The display should be big enough to visualize complex information in order to fulfill the task of focused interaction, while at the same time a long battery life is required of the device.

A bracelet-typed device can fulfill all the requirements stated above. It is slim and doesn't encumber you, so it can be worn on the arm all day.