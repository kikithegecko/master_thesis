\chapter{Evaluation}
\label{chap:evaluation}
\section{Impressions of the Device and its Interactivity}
Instead of a classical user study, the setup of interactive bracelet and lamp was presented to three participants in an informal structured interview. At first, the concept of casual interaction was explained to the participants and they were encouraged to reflect on the concept and think about interaction design for the scenario. Afterwards, the developed bracelet and its interaction was presented to the participants. There was no task to be completed, instead the users' approach towards the new device was observed and assisted. All participants were male students and had a background in Computer Science.

Users found interacting with the bracelet enjoying and quickly learned the basic functionality. Especially the hue slider during the precise color input was fascinating to the participants and they enjoyed playing with that feature. However, the function often triggered unintended while using the bracelet. The gesture input was also visibly enjoyed by the participants.

Tapping on the back of the bracelet to trigger certain functions was slightly misunderstood by some users. They didn't find the exact spot to tap on or tapped too gently on the device. This resulted always in a misconception about the device and the fact that the taps are triggered by an accelerometer instead of a touch surface. After explaining the concept to them, the interaction succeeded better. An improved and optimized configuration for the tap detection settings (cf. section \ref{sec:config}) could eliminate this problem if it succeeds to increase the tap sensitivity while maintaining the low rate of unintended activation through other interactions.

After the interview, the participants were asked to anonymously rate the overall experience with the device and its interactions on a five-point Scale from ``not good at all'' (1) to ``very good'' (5). An average rating of 4 was given.

\section{Lessons Learned}
When designing a wearable electronic device for the wrist, the most important thing is to create an appealing shape and texture, so the user would have no problems in wearing the device for a longer time. Printed material has a rough texture that is uncomfortable to the skin and can even cause skin irritations if it scrubs for a longer amount of time. Cast silicone is much more skin friendly, and in addition it is a flexible material which allows for mounting parts without the need for additional adhesives, since simply adding pockets that hold the parts in under little to medium tension is often the best solution.

However, when the use of adhesives is not avoidable by any means, cast silicone is a very tricky material. Liquid glue and adhesive tape do not hold reliably, especially when the part to be attached would be under frequent tension. Hence, designing the wearable to be manufactured with only one cast is much desired. This should be especially taken into consideration when designing closing mechanisms for silicone wearables.

While designing the electronic components, saving space has high priority. This inevitably leads to choosing surface mounted components. Some of those small packages are still possible to solder by hand, but very small packages for integrated circuits like the accelerometer used in this thesis usually lead to the need of a reflow soldering process. If there is no such equipment available, the use of very small components should be limited, since ordering populated boards at a factory usually takes a couple of days for them to arrive. The same holds for prototype \ac{PCB}s: Parts with small footprints lead to small lines on the board. It was perceived that the ability to mask and etch prototype boards featuring small footprints and thin lines somehow correlates with the ability to solder the respective components manually.

Choosing small sized parts leads to less required \ac{PCB} space. However, the components still need to be routed, which becomes more complicated when the board space shrinks and eventually leads to the need for double-layered \ac{PCB}s. Those, however, are complicated to fabricate with the low-cost masking and etching process, they either require advanced equipment like a CNC mill, or the involvement of a fabrication service. This needs to be considered when deciding for a double-layered \ac{PCB}.

Embedded prototyping boards like the Arduino device family or the Teensy board can be easily extended by stacking custom extension boards on top of them. As convenient as this may seems, it should be taken into account that extension boards are usually connected to standard spaced pin headers, which introduce extra height to the electronics piece that feels very bulky when attached to a wrist. A workaround to this problem was soldering an IC socket on the Teensy board instead of standard pin headers to save one or two millimeters in component height. The extension board featured manually shortened pin connectors. Such a setup should only be considered for the final iteration of the hardware, since the combination is hard to disassemble once the extension board has been firmly pressed into the socket. Even with this workaround, the board still feels a little clumsy and requires careful design of the casing.

When implementing code on an embedded platform, the processor's architecture should be taken into consideration, especially when a computation-intensive algorithm is planned to run on the device. The Teensy's processor did not have a floating point computing unit, so implementing the gesture recognition algorithm which relies heavily on trigonometry would have needed extensive modifications like using precalculated tables for trigonometric functions. Since those adjustments would have exceeded the given time frame for this thesis, the gesture recognition code was outsourced to a stationary PC to which the bracelet is connected for power supply.